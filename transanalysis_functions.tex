%% LyX 2.0.6 created this file.  For more info, see http://www.lyx.org/.
%% Do not edit unless you really know what you are doing.
\documentclass[english]{article}
\usepackage[T1]{fontenc}
\usepackage[utf8]{luainputenc}
\usepackage{babel}
\begin{document}

\title{Transanalysis IPython Notebooks and Functions}


\date{16 December 2013}


\author{John Thomas Sauls}

\maketitle

\subsection*{IPython Notebooks}
\begin{itemize}
\item import\_block - Holds some common start up stuff, import models, loading
transcription data, etc.
\item reaction\_connectedness - Looks at either iJO or the core model and
the minspans within them. Finds minspans each reaction is in, and
the size of the minspans. FPKM averages for minspans can be calculated
while excluding highly connected reactions
\item minspan\_shifts - Explores how the average FPKM value changes per
minspan under the different conditions we have. Also finds most differentially
regulated minspans under aerobic vs anaerobic conditions and maps
this with Visbio.
\item core\_minspans - Looks at just the core minspans as calculated from
the core model. Still uses iJO in order to associate the FPKM data.
\item core\_flux - This explores how the FBA solution for the core model
stacks up with the FPKM data, in the context of minspans. 
\item ijo\_flux - Same thing for iJO1366. Note this script and in the one
above, flux per minspan is calculated with the projection, or just
a simple averaging.
\item find\_bottlenecks - Solves iJO under pFBA and E-flux, and sees where
individual reactions are over constrained. Is a precursor for the
relax\_eflux script.
\item relax\_eflux - This script finds the bottlenecks as above, and then
relaxes those bounds in a recursive manner so E-flux does not over
constrain the model.
\item compare\_fpkms\_reactions - Combining FPKM data per reaction, compares
two FPKM datasets to see how correlated they are under similar and
disparate conditions. 
\item microarray\_vs\_fpkm - Does a comparison by reactions like above,
but uses data from the microarrays as well. 
\end{itemize}

\subsection*{transAnalysis Functions}

These functions have descriptions, as well as input and outputs in
their docstring.
\begin{itemize}
\item make\_gene\_fpkm\_dict - Makes a dictionary from an FPKM tracking
file with genes as keys and FPKM levels as values
\item make\_reaction\_fpkm\_dict - Using the GPRs associations, makes a
dictionary with reactions as keys and combined fpkm values for the
related genes as values.
\item make\_minspan\_list - Creates a list of minspans from the excel file
which contains them. Either for iJO or core model as tested.
\item make\_minspan\_fpkm\_list - Averages FPKM values of the reactions
for each minspan. Can choose a cut off value to exclude highly connected
reactions.
\item rank\_reactions - Rank reactions based on FPKM value as computed through
the GPRs.
\item fpkm\_comparison - Takes two FPKM filenames, a model, and a list of
minspans and outputs a graph and correlation values for how the minspans
line up in terms of their FPKM values.
\item Eflux - Perform E-flux as per the Colijn et al. paper. Modified form
the script the Danish guys sent.
\item relax-bounds - Takes a model with an E-flux solution and recursively
releases the bounds to find over-constrained reactions.
\item make\_minspan\_k\_dict - A dictionary with minspan IDs as keys and
the involved reactions in a list as values.
\item make\_reaction\_k\_dict - A dictionary with reaction IDs as keys and
involved minspans in a list as values.
\item make\_reaction\_k\_dict1 - Same as above, but draws the reaction list
from the minspans, so only reactions that are involved in minspans
end up in the dictionary. 
\item run\_pfba - Solve the model using pFBA, modified from Teddy's script.
\item make\_model\_irrev - Used for run\_pfba\end{itemize}

\end{document}
